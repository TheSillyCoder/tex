\documentclass[10pt]{article}

\usepackage[T1]{fontenc}
\usepackage{geometry}
\usepackage{graphicx}
\usepackage{amssymb}

\title{HU1102}
\author{Debayan Sarkar}
\date{}

\geometry{a4paper, margin=1in}
\setlength{\parindent}{0pt}


\begin{document}
            \par\textbf{IISER Kolkata} \hfill \textbf{HU1102}
        \vspace{3pt}
        \hrule
        \vspace{3pt}
        \begin{center}
                \LARGE{\textbf{Sociology Notes}}
        \end{center}
        \vspace{3pt}
        \hrule
        \vspace{3pt}
        Debayan Sarkar, \texttt{22MS002}\hfill\today
        \vspace{20pt}
        \begin{center}
                \LARGE{\textbf{Max Weber (1864 - 1920)}}
        \end{center}
        \vspace{20pt}

        \textbf{Weberian Philosophy}
        \begin{enumerate}
        \item
        Like Marx, Weber noted that one of the most important historical transformations was the rationalisation of economic structures to produce modern capitalism. By rationalisation he meant, the process of making life more efficient and predictable by wringing out individuality and spontaneity in life. 
        \item
        In the protestant ethic and the spirit of capitalism, Weber characterised rational capitalism as the most fateful force in our modern life. 
        \item
        Weber's studies of modern rational capitalism were important responses to Marx's analysis of capitalism. Weber had a great admiration for Marx's brilliant constructions but he rejected the elevation of 'material factors' as absolute and being turned into common denominator of causal explanations. Both Weber and Marx sought to discover the historical causal relationships that had lead to the current state of modern society, but Weber did not claim as Marx did, that material factors or economic interests could explain every aspect of social reality.
        \item
        Ideas in Weber's view, especially religious ideas were important components of practical action. No economic ethic has ever been determined solely by religion.
        \item
        Weber's interest in modern rational capitalism was reflected in many of his analysis. The context, both personal and theoretical, for all these analysis was Weber's understanding that modern society was increasing a place in which the transcendental world of gods was giving way to science and the rational calculations of social actions.
        \item
        The progression from faith to secular pragmatism was also mirrored in and facilitated, by the triumph of western science. The spiritually motivated search for authentic 'truth's had fueled the development of science. But the development of science produced the search for efficient means-ends calculation and mastery over the natural world, which displaced the 'religious monopoly on truth' in the modern world.
        \item
        Weber pointed out that the belief in the value of scientific truth is a product of certain cultures and is not a product of man's original nature. More specifically, the rationality of western capitalism depended on the calculability of the most important technical factors. In other words in depended on the sciences especially those based on mathematics and exact rational experiment. 
        \end{enumerate}
        
\end{document}
