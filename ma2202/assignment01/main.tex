\documentclass[10pt]{scrartcl}

\usepackage[T1]{fontenc}
\usepackage{geometry}
\usepackage{graphicx}
\usepackage{amssymb}
\usepackage{amsmath}
\usepackage{enumitem}
\usepackage[english]{babel}
\usepackage{amsthm}
\usepackage{fancyhdr}
\geometry{a4paper, margin=0.8in}
\pagestyle{fancy}
\lhead{MA2202 - Assignment 01 Solutions}
\rhead{Debayan Sarkar \texttt{22MS002}}
\everymath{\displaystyle}
\theoremstyle{definition}
\newtheorem{exercise}{Exercise}
\newenvironment{solution} {\begin{proof}[\normalfont \textbf{Solution}]} {\end{proof}}
\newcommand{\nn}{\mathbb{N}}
\newcommand{\rn}{\mathbb{R}}
\newcommand{\q}{\mathbb{Q}}
\newcommand{\p}{\mathcal{P}}
\newcommand{\z}{\mathbb{Z}}
\title{MA2202 - Probability I}  
\subtitle{Assignment 1 Solutions}
\author{Debayan Sarkar \\ \texttt{22MS002} \\ Group B}
\date{\today}

\geometry{a4paper, margin=1in}
\setlength{\parindent}{0pt}
\begin{document}
\maketitle
\begin{exercise}(10 Points)

    Show that if you keep on throwing a fair die, the probability of eventual occurence of the outcome six is 1.
\end{exercise}
\begin{solution}
    If we throw a fair die till we find a six, the possible outcomes are $$\Omega = \{\text{Get a six after a finite number of tries} \} \cup \{ \text{Never get a six} \}$$
    Let $A$ be the event of eventually getting a six after a finite number of tries, and let $B$ be the event of never getting a six
    and let $A_n$ be the event of getting a six after the $n^{th}$ try, $\forall n\in \nn$. Now, since the die is fair, for the event 
    $A_n$, for each of the the first $n-1$ throws, the probability of not getting a six is $\cfrac{5}{6}$, and the probability of getting a six
    on the $n^{th}$ throw is $\cfrac{1}{6}$. Hence, \fbox{$P(A_n) = \cfrac{1}{6}\cdot\left(\cfrac{5}{6}\right)^{n-1}$}

    Now, clearly $A = \cup_{n=1}^{\infty}A_n$ and $A_i \cap A_j = \phi ~ \forall i \neq j$ then using the axioms of probability we have, 
    \begin{align*}
        P(A) &= P\left(\cup_{n=1}^{\infty} A_n\right) \\ 
             &= \sum_{n=1}^{\infty} P(A_n) \\ 
             &= \sum_{n = 1}^{\infty} \cfrac{1}{6} \cdot \left(\cfrac{5}{6}\right)^{n-1} \\ 
             &= \cfrac{1}{6}\cdot \sum_{n = 1}^{\infty} \left(\cfrac{5}{6}\right)^{n-1} \\ 
             &= \cfrac{1}{6} \cdot \cfrac{1}{1- \cfrac{5}{6}} \\ 
             &= \cfrac{1}{6}\cdot 6 \\ 
             &= 1
    \end{align*}
    Hence, the probability of eventually getting a six is 1.
\end{solution}
\clearpage
\begin{exercise}(10 Points)

    Suppose $(\Omega, \mathcal{E}, P)$ be a given probablity space and $A_1, A_2, \dots ,A_n \in \mathcal{E}$. Then show that 
    $$P(A_1 \cup A_2 \cup \dots \cup A_n) = \sum_{i=1}^{n}P(A_i) - \sum_{1\leq i < j\leq n}^{n}P(A_i \cap A_j) + \dots + (-1)^{n - 1}P(A_1 \cap \dots \cap A_n)$$
\end{exercise}
\begin{solution}
    We shall prove this via induction on $n$ i.e. the number of events. For the base case when $n=2$, observe that,
    $A_1 = (A_1 \cap A_2) \cup (A_1 cap A_2^c)$. Where $A_2^c = \Omega \setminus A_2$. Since $A_1 \cap A_2$ and 
    $A_1 \cap A_2^c$ are mutually exclusive, using the axioms of probability we have 
    \begin{equation}
        P(A_1) = P(A_1 \cap A_2) + P(A_1 \cap A_2^c)
    \end{equation}
    Also, 
    \begin{align*}
        P(A_1 \cup A_2) &= P(A_1 \cap A_2^c) + P(A_2) \tag{$(A_1 \cap A_2^c) \cap A_2 = \phi$} \\ 
                        &=P(A_1) + P(A_2) - P(A_1 \cap A_2) \tag{from (1)}
    \end{align*}
    Let us assume that this statement holds for some $n=k$. We wish to show that it holds for $n = k + 1$. 
    \begin{align}
        P(\cup_{i = 1}^{k + 1}) &= P(\cup_{i = 1}^{k}A_i\cup A_{k + 1}) \nonumber\\ 
                                       &= P(\cup_{i = 1}^{k}A_i) + P(A_{k + 1}) - P(\cup_{i = 1}^{k}A_i \cap A_{k + 1}) \tag{Base Case} \\
                                       &=  \sum\limits_{i = 1}^{k} P(A_i) - \sum\limits_{1\leq i < j\leq k}^{n}P(A_i \cap A_j) + \dots + (-1)^{k - 1}P(\cap_{i = 1}^{k}{A_i}) + P(A_{k + 1}) - P(\cup_{i = 1}^{k}A_i \cap A_{k + 1}) \tag{Induction Hypothesis}\\ 
                                       &=  \sum\limits_{i = 1}^{k + 1} P(A_i) - \sum\limits_{1\leq i < j\leq k}^{n}P(A_i \cap A_j) + \dots + (-1)^{k - 1}P(\cap_{i = 1}^{k}{A_i}) - P(\cup_{i = 1}^{k}A_i \cap A_{k + 1})
    \end{align}
    Now, we also have
    \begin{align*}
        P(\cup_{i = 1}^{k} A_i \cap A_k) &= P(\cup_{i = 1}^{k} (A_i \cap A_k)) \tag{Distributivity of unions and intersections}\\ 
                                         &= \sum\limits_{i = 1}^{k}P(A_i \cap A_{k+1}) - \dots + (-1)^{k- 1}P(\cap_{i = 1}^{k + 1}A_i) \tag{Induction Hypothesis}
    \end{align*}
    Substituting this into (2) we get,
    \begin{align*}
        P(\cup_{i = 1}^{k + 1}) = \sum\limits_{i = 1}^{k + 1}P(A_i) - \sum\limits_{1 \leq i < j \leq k + 1}^{k + 1}P(A_i \cap A_j) + \dots + (-1)^{k}P(\cap_{i = 1}^{k + 1} A_i)
    \end{align*}
    Hence, the statement is true for $n = k + 1$. This proves our claim.

\end{solution}
\begin{exercise}(10 Points)

    To the choice of each $n \in \nn$, could you assign a probability $P(n) > 0$ such that the following conditions hold?
    \begin{enumerate}[label={(\alph*)}]
        \item $P(m) \neq P(n)$ for all $m, n \in \nn$
        \item The probability of choosing an odd positive integer is the same as the probabilty of choosing an even integer.
    \end{enumerate}
    Justify your answer!
\end{exercise}
\begin{solution}
    Let us define $P: \nn \to [0, 1]$ as,
    $$
    P(n) = 
    \begin{cases}
        \left(\cfrac{1}{2}\right)^{\cfrac{n}{2} + 1} ~ \text{ if } n \text{ is even} \\ 
        \left(\cfrac{1}{3}\right)^{\cfrac{n + 1}{2}} ~ \text{ if } n \text{ is odd}
    \end{cases}
    $$
    Then, clearly $P(m) \neq P(n) ~ \forall m \neq n$ since 2 and 3 are prime numbers and thus, condition (a) is satisfied.
    Furthermore, Let $E := \{2k : k \in nn\}$ and $O := \{2k - 1 : k \in \nn \}$
    $$P(E) = \sum_{k = 1}^{\infty} P(2k) = \sum_{k = 1}^{\infty} \left(\cfrac{1}{2}\right)^{k + 1} = \cfrac{1}{2}$$
    and, 
    $$P(O) = \sum_{k = 1}^{\infty} P(2k - 1) = \sum_{k = 1}^{\infty} \left(\cfrac{1}{3}\right)^{k} = \cfrac{1}{2}$$

    Thus, it also satisfies condition (b).
\end{solution}
\begin{exercise}(10 Points)

    Seven students of IISER Kolkata went to participate in an event at IISER Mohali. They booked
    AC 3-tier tickets from Howrah to Chandigarh in Netaji Express, which has three AC 3-tier
    coaches. Every such coach has eight coupes (i.e. compartments), each coupe containing eight
    berths. If the berths were allocated randomly, find the probability of at least two among the
    seven students being allocated berths in the same coupe.
\end{exercise}
\begin{solution}
    Since each coupe has more than 7 berths, all can be considered to be equally likely. Then, we have a total of 24 coupes, and 
    7 people. The probability that atleast two of them will be allocated berths in the same coupe is 

$P(\text{Atleast two in the same coupe}) = 1 - P(\text{No two in the same coupe}) = 1 - \cfrac{^{24}P_7}{24^7} \approx 0.62$
\end{solution}
\begin{exercise}(10 Points)

    Carrom is played with a red, nine black and nine white coins (and a striker) on a square board
    with a pocket in each corner. If all these coins are scattered randomly (none of them being in
    any pocket) on a 29 inch $\times$ 29 inch carrom board, show that the probability of at least two of
    the coins being less than three inches apart is greater than $\cfrac{1}{2}$.
\end{exercise}
\begin{solution}
    We have a total of 19 coins, and a 29 inch $\times$ 29 inch carrom board. We will assume the coins to be point objects, 
    and the carrom board to be a square of dimensions $29 \times 29$. Hence, it suffices to show, that if we place 
    19 points on a square of the given dimension, the probability that atleast 2 of them will be less than 3 inches apart is 
    more than 1/2. 

    Let us divide the given square, into smaller squares of dimension $\cfrac{29}{14} \times \cfrac{29}{14}$, with sides parallel to the the sides of the original square. 
    Then, the square has now been equally divided into 196 smaller squares. Observe that, the length of the diagonals of the small squares is 
    $\cfrac{29}{14}\sqrt{2} \approx 2.93 < 3$. Hence, if we have any two points in one of these squares, the distance between them is less than 3.

    The probablity that atleast 2 out of the 19 coins will be in the same square can calculated as discussed in class, as 
    $1 - \cfrac{^{196}P_{19}}{196^19} \approx 0.59 > \cfrac{1}{2}$.

    Hence, the probability that atleast 2 of the coins will be less than 3 inches apart is greater than $\cfrac{1}{2}$.
\end{solution}
\end{document}

