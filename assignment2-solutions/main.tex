\documentclass[10pt]{article}

\usepackage[T1]{fontenc}
\usepackage{geometry}
\usepackage{graphicx}
\usepackage{amssymb}
\usepackage{amsmath}
% \usepackage{enumitem}
\usepackage[english]{babel}
\usepackage{amsthm}
% \usepackage[hidelinks]{hyperref}

\newcommand{\nn}{\mathbb{N}}
\newcommand{\rn}{\mathbb{R}}
\newcommand{\q}{\mathbb{Q}}
\newcommand{\p}{\mathcal{P}}
\newcommand{\z}{\mathbb{Z}}
\title{MA1201 - Assignment 2 Solutions}
\author{Debayan Sarkar}
\date{May 28, 2023}

\geometry{a4paper, margin=0.8in}
\setlength{\parindent}{0pt}
\begin{document}
\maketitle
\begin{enumerate}
    \item Prove that the collection of all finite subsets of $\nn$ is countable.

	    \textbf{Solution : }Let $N_n$ denote the set of all subsets of $\nn$ with $n$ elements.
	    Then consider the function $f : N_n \to \nn$ defined as, 
	    $$f(S) = \sum_{n\in S} 10^{n} \,\,\, \forall S \in N_n$$
	    $f$ can be shown to be injective. Since $f$ is an injection from $N_k$ to $\nn$ which is a countable set, 
	    $N_k$ is also countable. The set of all finite subsets of $\nn$,
	    $$X := \bigcup_{n \in \nn} N_n$$ is the countable union of countable sets, and hence is countable. \qed

	    \textbf{Note : }The set $X$ must not be confused with $\p(\nn)$. Even though both of them may seem similar as $n$ grows larger and larger, 
	    it's easy to see that $\nn \in \p(\nn) \text{ but } \nn \notin X$. Hence, $X \neq \p(\nn)$
    \item Prove that $\nn \times \nn$ is countable.

    \textbf{Solution : }Let us define a function $f : \nn \times \nn \to \nn$ as $$f((x, y)) = 2^x 3^y \,\,\, \forall (x, y) \in \nn \times \nn$$ Let $(x_1, y_1) , (x_2, y_2) \in \nn \times \nn$ such that, $f((x_1, y_1)) = f((x_2, y_2))$. Then, 
    \begin{align*}
        f((x_1, y_1)) = f((x_2, y_2))
        \Rightarrow &2^{x_1}3^{y_1} = 2^{x_2}3^{y_2} \\
        \Rightarrow &2^{x_1 - x_2} = 3^{y_2 - y_1}\\
        \Rightarrow &x_1 = x_2 \And y_1 = y_2 \tag{Since 2 and 3 are prime numbers}
    \end{align*}
    Hence, $f$ is injective. Since, $f$ is an injection from $\nn \times \nn$ to $\nn$, and $\nn$ is countable, $\nn \times \nn$ is also countable. \qed
    \item Show that the set $F = \{ p + q\sqrt{2}: p,q \in \q \}$ is countable.
    
    \textbf{Solution : }Let us define $f: \q \times \q \to F$ as, $$f((x, y)) := x + y\sqrt{2} \,\,\, \forall (x, y) \in \q \times \q$$ Now we claim that $f$ is a bijection. Let $(a_1, b_1), (a_2, b_2) \in \q \times \q$ such that $f((a_1, b_1)) = f((a_2, b_2))$. Then, 
    \begin{align*}
        f((a_1, b_1)) = f((a_2, b_2) )
        \Rightarrow &a_1 + b_1\sqrt{2} = a_2 + b_2\sqrt{2} \\
        \Rightarrow &(a_1 - a_2) = (b_2 - b_1)\sqrt{2} \\
        \Rightarrow &a_1 = a_2 \And b_1 = b_2 \tag{$\q$ is closed under addition} \\ 
        \Rightarrow &(a_1, b_1) = (a_2, b_2)
    \end{align*}
    Hence, $f$ is injective. Now, let $z \in F$ be arbitrary. Then, $z = p + q\sqrt{2}$ for some $p, q \in \q$. Then, $f((p,q)) = p + q\sqrt{2} = z$. Hence $f$ is onto. Thus, $f$ is bijective. Hence, $\q \times \q \sim F$. Since $\q \times \q$ is countable, $F$ is also countable. \qed
    \item Prove that the set of all polynomials of degree $\leq$ 3 with integer coefficients is countable.

    \textbf{Solution : } Let $S$ be the set of all polynomials of degree $\leq$ 3 with integer coefficients. Let us define $f : \z \times \z \times \z \times \z \to S$  as, $$f((a, b, c, d)) := P(x) \,\,\, \forall (a, b, c, d) \in \z \times \z \times \z \times \z$$ Where $P(x) = ax^3 + bx^2 + cx + d \,\,\, \forall x \in \rn$. We claim that $f$ is a bijection. Let $(a_1, b_1, c_1, d_1), (a_2, b_2, c_2, d_2) \in \z \times \z \times \z \times \z$ such that $f((a_1, b_1, c_1, d_1)) = f((a_2, b_2, c_2, d_2)) \Rightarrow a_1x^3 + b_1x^2 + c_1x + d_1 = a_2x^3 + b_2x^2 + c_2x + d_2 \,\,\, \forall x \in \rn$.
    Since the polynomials are equal for all values of $x$, the coefficients must be equal i.e. $(a_1, b_1, c_1, d_1) = (a_2, b_2, c_2, d_2)$. Hence, 
    $f((a_1, b_1, c_1, d_1)) = f((a_2, b_2, c_2, d_2)) \Rightarrow (a_1, b_1, c_1, d_1) = (a_2, b_2, c_2, d_2)$ thus, $f$ is injective. Now, let $P(x) \in S$ defined as  $P(x) = ax^3 + bx^2 + cx + d \forall x \in \rn$ be arbitrary. Then let $z = (a, b, c, d) \in \z \times \z \times \z \times \z$. Then, clearly, $f(z) = P(x)$. Thus, $f$ is surjective. Hence, $f$ is bijective. Thus, $\z \times \z \times \z \times \z \sim S$. Since $\z \times \z \times \z \times \z$ is countable, $S$ is also countable. \qed
    \item Prove that the sets $(0, \infty)$ and $\rn$ are equipotent.
    
    \textbf{Solution : }Let us define $f : (0, \infty) \to \rn$ as, $$f(x) := \ln(x) \,\,\,\, \forall x \in (0, \infty)$$ Now, we claim that $f$ is a bijection. 

    Let $x_1, x_2 \in (0, \infty)$ such that $f(x_1) = f(x_2)$. 
    \begin{align*}
        f(x_1) = f(x_2)
        \Rightarrow &\ln(x_1) = \ln(x_2) \\
        \Rightarrow &e^{\ln(x_1)} = e^{\ln(x_2)} \\
        \Rightarrow &x_1 = x_2
    \end{align*}
    Hence, $f$ is injective. Let $y \in \rn$ be arbitrary. Then let us define $x := e^y \in (0, \infty)$. Then, $\ln(x) = \ln(e^y) = y$. Hence, $f$ is surjective. Thus, $f$ is a bijection. This proves our claim. Since $f$ is a bijection, the sets $(0, \infty)$ and $\rn$ are equipotent. \qed
    \item Prove that if $A$ and $B$ are countable then $A \times B$ is countable. In general for every $n \in \nn$ if $A_1, A_2, \dots , A_n$ are countable then $A_1 \times A_2 \times \dots \times A_n$ is countable.
    \item Prove or disprove that $A_1 \times A_2 \times \dots$ is countable, where $A_i$ is a countable set.

	    \textbf{Hint : }This set will be uncountable (if each $A_i$ has atleast 2  elements). Use Cantor's diagonal argument. (See Problem 13) 
    \item Prove or disprove that countable union of countable sets in countable.
    \item Prove that the set $A = \{ a \in \rn : \exists a^4 + pa^3 + qa^2 + ra + s = 0 \}$ is countable.
    
    \textbf{Solution : }We know that $\z^4$ is countable. Let $\z^4 = \{z_1, z_2, \dots \}$. Then, for some $n \in \nn$ let us construct $A_n$ as, $$A_n = \{a \in \rn : pa^3 + qa^2 + ra + s = 0, (p,q,r,s) = z_n \in \z^4 \}$$ Since, $A_n$ can have atmost 4 elements, $A_n$ is finite and hence countable for all $n \in \nn$. Also, $$A = \bigcup_{n\in\nn} A_n$$ Hence, $A$ is also countable since the countable union of countable sets is countable. \qed
    \item Let $A$ be a finite set. Prove that the set of all sequences of elements in $A$ of finite length is countable.

	    \textbf{Solution : }Each finite sequence of elements in A of length $n \in \nn$ can be written as 
	    an element of $A^n$. Since $A$ is countable, $A^n$ is also countable $\forall n\in \nn$
	    Hence, the set of all sequences of elements in A of finite length, 
	    $$ X:=\bigcup_{n \in \nn} A^n$$ is the countable union of countable sets and hence, is countable. \qed
    \item Prove that if $X$ is countable and $f : X \to Y$ is a surjective function then $Y$ is countable.

	    \textbf{Solution : } Look at the solution of 23(c).
    \item Let $A$ be an infinite set. If there is an infinite sequence in which each element of $A$ appears at least once, then show that $A$ is countable.
    
    \textbf{Solution : } Since there is an infinite sequence in which each element of $A$ appears at least once, $\exists f : \nn \to A$ such that $f$ is surjective. Since $\nn$ is countable, $A$ must also be countable. \qed    
    \item Prove that the set of all functions from $\nn$ to $\nn$ is uncountable.

    \textbf{Solution : }Let $S$ be the set of all functions from $\nn$ to $\nn$. We claim that $S$ is uncountable. Let us assume to the contrary that, $S$ is countable. Then, the set $S$ can be written as $$S = \{f_1, f_2, \dots \}$$ Now, let us construct $f : \nn \to \nn$ as, 
    \begin{equation*}
        f(n) = 
        \begin{cases}
            1 & \text{if } f_n(n) \neq 1 \\
            2 & \text{if } f_n(n) = 1
        \end{cases}
    \end{equation*}
    Then, $f(n) \neq f_n(n) \,\,\, \forall n \in \nn$, i.e. $f \neq f_n \,\,\, \forall n \in \nn$ Hence, $f \notin S$. This is a contradiction since, $S$ is the set of all functions from $\nn$ to $\nn$. Hence, the set $S$ is uncountable. \qed

    \textbf{Note : }This approach of constructing an element by making all the diagonal elements unequal  to prove the uncountability of a set is known as 
    Cantor's diagonal argument. This approach will be used multiple times throughout the solutions.
    \item Prove that the set of all decreasing functions from $\nn$ to $\nn$ is countable.

	    \textbf{ Solution : }Let the set of all decreasing functions from $\nn$ to $\nn$ be $A$ and, let $f \in A$ be arbitrary.
	    As a consequence of well-ordering principle, $\exists m,n \in \nn$ such that $\forall k \in \nn \text{ with }k \geq n$, 
	    $f(k) = m$ i.e. the function becomes a constant function after $n$. We can uniquely
	    represent this function as an ordered tuple $(f(1), f(2), f(3), \dots f(n)) \in \nn^n$.
	    Let $A_n$ be the collection of all decreasing functions that become constant after some $n \in \nn$. Then, $A_n \subseteq \nn^n \Rightarrow A_n \text{ is countable.} \,\,\, \forall n \in \nn$

	    Then, the collection of all decreasing functions from $\nn$ to $\nn$
	    $$ A = \bigcup_{n \in \nn}A_n$$
	    is the countable of union of countable sets and hence countable. \qed


    \item Let $X$ and $Y$ be two nonempty finite sets. Then, is the set of all functions from $X$ to $Y$ (i) finite (ii) countably infinite (iii) uncountable? Give justifications

	    \textbf{Solution : }Let's assume that the set $X$ has $m$ elements and the set $Y$ has $n$ elements. Then, For each element in $X$, there are $n$ possible elements it can map to. Hence, the total number of functions will be $n^m$. Hence, the set of all functions from $X$ to $Y$ is finite. \qed
    \item Prove that a set is infinite iff it is bijective with a proper subset of itself.

	    \textbf{Solution : }Let's assume $S$ is an infinite set. Let $A \subset S$ be countable.
	    We know such a subset exists because, every infinite set has a countable subset. Then,
	    the set $A$ can be written as $A = \{ a_1, a_2, \dots \}$. Let us split thus sets into two sets $A_1$ and $A_2$, as
	    $A_1 = \{ a_1, a_3, \dots \}$ and $A_2 = \{ a_2, a_4, \dots \}$. Now, let us consider the function $f: S \to S \setminus A_1$ defined as,
	    \begin{equation*}
		    f(x) = 
		    \begin{cases}
			    x & \text{ if } x \in S \setminus A \\
			    a_{2n} & \text{ if } x = a_n \in A
		    \end{cases}
	    \end{equation*}
	    $f$ can be shown to be a bijection and hence, $S \sim S\setminus A_1 \subset S$

	    Now, let us assume that $S$ is equipotent to a proper subset of itself. We shall show that
	    $S$ is ininite. Let's assume to the contrary that, $S$ is finite. Then, Let $S$ have $n \in \nn$ elements. Then
	    any proper subset of $S$ has $m \in \nn$ elements with $m < n$. Then, it is impossible to find a bijection between them
	    since, for a bijection to exist between two finite sets, the number of elements in the sets must be equal. But, this is a contradiction.
	    Hence, $S$ must be inifinte. \qed


    \item Prove that if $A \cap B = \phi$ then $\p(A \cup B) \sim \p(A) \times \p(B)$

	    \textbf{Solution : }Consider the function $f : \p(A \cup B) \to \p(A) \times \p(B)$ defined as,
	    $$f(S) = (\{ x : x \in S \cap A \},\{ y : y \in S \cap B \}) \,\,\, \forall S \in \p(A \cup B)$$
	    Let $S_1, S_2 \in \p(A \cup B)$ such that $S_1 \neq S_2$. WLOG we can assume that, $\exists x \in S_1 \text{ such that } x \notin S_2$

	    \textbf{Case 1 }$x \in A$ 

    $x \in A \Rightarrow x \in S_1 \cap A \text{ and } x \notin S_2 \cap A$
    Hence, $\{a : a \in S_1 \cap A \} \neq \{ b : b \in S_2 \cap A \} \Rightarrow f(S_1) \neq f(S_2)$

	    \textbf{Case 2 }$x \in B$ 

    $x \in B \Rightarrow x \in S_1 \cap B \text{ and } x \notin S_2 \cap B$
    Hence, $\{a : a \in S_1 \cap B \} \neq \{ b : b \in S_2 \cap B \} \Rightarrow f(S_1) \neq f(S_2)$

    Hence, $f$ is injective. Now, let $(S_A, S_B) \in \p(A) \times \p(B)$ be arbitrary. Then, $S_A \subseteq A$ and $S_B \subseteq B$.
    Let $S = S_A \cup S_B$. Then, $S \cap A = S_A \cup S_B \cap A = S_A \cup \phi = S_A (\text{ since } A \cap B = \phi)$. Similarly $S \cap B = S_B$.
    Then, 
    \begin{align*}
	f(S) &= (\{ x : x \in S \cap A \}, \{ y : y \in S \cap B \}) \\
     	&= (\{x : x \in S_A\}, \{y : y \in S_B\})\\
	&=(S_A, S_B)
    \end{align*}
    Hence, $f$ is onto. Thus $f$ is a bijection and thus, $\p(A \cup B) \sim \p(A) \times \p(B)$ \qed

    \item Is the set of all infinite sequences of 0's and 1's finite, countably infinite or uncountable? Give justification.

    \textbf{Solution : }Let $S$ be the set of all infinite sequences of 0's and 1's. We claim that $S$ is uncountable. Let us assume to the contrary that, $S$ is countable. Then, the set $S$ can be written as, $$S = \{ a_1, a_2, a_3, \dots \}$$ Also, $i^{th}$ element in $S$ can be indexed as well like this, $$a_i = a_{i1} a_{i2} a_{i3} \dots \text{ where } a_{ij} \in \{0, 1\} \,\,\, \forall i,j \in \nn$$ 
    Now, let us construct $a_0$ as ,
    \begin{equation*}
        a_{0n} = 
        \begin{cases}
            1 & \text{if }a_{nn} = 0 \\
            0 & \text{if }a_{nn} \neq 0\\
        \end{cases}
        \,\,\, \forall n \in \nn
    \end{equation*}
    Clearly $a_0 \neq a_n \,\,\, \forall n \in \nn$. Hence, $a_0 \notin S$. But this is a contradiction since $S$ is the set of all infinite sequences of 0's and 1's. Hence, our assumption is wrong and the set $S$ is uncountable. \qed
    \item Give an example of two sets $A$ and $B$ such that $B \subset A$ and $B$ is bijective with $A$ but $B \neq A$.

    \textbf{Solution : }Let $A := \nn$ and $B := \{2n : n \in \nn \}$. Then, clearly $B \subset A$ and $B \neq A$ but the function $f : A \to B$ defined as $f(a) = 2a \,\,\, \forall a \in A$ is bijective i.e. $A \sim B$.
    \item Prove that if there exists an injective function from $(0, 1)$ to a set $A$ then the set A is uncountable.

	    \textbf{Solution :} Let $f : (0, 1) \to A$ be injective. Since, $(0, 1)$ is uncountable,
	    $A$ is uncountable.

	    Now we shall prove that $(0, 1)$ is uncountable. Let us assume to the contrary that $(0, 1$ is countable.
	    Let $X = (0, 1)$. Then, the set $X$ can be written as, $X = \{ x_1, x_2, \dots \}$. Also, since $x_i \in (0, 1)$, each $x_i$ can be written as,
	    $$x_i = 0.x_{i1}x_{i2} \dots $$ where $x_{ij}$ refers to the $j^{th}$ digit after the decimal point in the decimal
	    expansion of $x_i$.
	    Now, let us construct $x_0$ as,
	    \begin{equation*}
		    x_{0n} = 
		    \begin{cases}
			    1 & \text{ if } x_{nn} \neq 1 \\
			    2 & \text{ if } x_{nn} = 1
		    \end{cases}
		    \forall n \in \nn
	    \end{equation*}
	    Then, $\forall n \in \nn \text{ we have, } x_{0n} \neq x_{nn} \Rightarrow x_0 \in x_n \Rightarrow x \notin B$. This is a contradiction. Hence, $B$ cannot be countable.
	    Hence the set $(0, 1)$ is uncountable.


    \item Suppose that $A \subseteq B$ then prove that 
    \begin{enumerate}
        \item $B$ is finite $\implies$ $A$ is finite.

		\textbf{Solution : }We shall prove this using induction on the number of elements in B.
		When the number of elements in $B$ is 1, The two subsets are $\phi$ and $B$ which are trivially finite. This is our base case. 
		
		Now, let's assume any subset of a set containing $n$ elements is finite. Let the set $B$ have $n + 1$ elements.
		Then, let $A \subseteq B$. If $A = B$, then $A$ is trivially finite since $B$ is finite. If $A \neq B$, then 
		$\exists x \in B \text{ such that } x \notin A$. Then, $A \subseteq B\setminus\{x\}$. The set $B\setminus \{x\}$ has 
		$n$ elements and hence, $A$ is again countable. Thus the statement is true for $n + 1$ whenever it's true for $n$. By the principle of
		mathematical induction, the statement holds true $\forall n \in \nn$.\qed
        \item $A$ is infinite $\implies$ $B$ is infinite.

        \item $B$ is countable $\implies$ $A$ is countable.

		\textbf{Hint :}First show that any subset of $\nn$ is countable. Then, use that fact to prove the given problem by 
		finding a bijection between a subset of $\nn$ and $A$.
        \item $A$ is uncountable $\implies$ $B$ is uncountable.
    \end{enumerate}
    \item Suppose $f : A \to B$ is injective then prove that 
    \begin{enumerate}
        \item $B$ is finite $\implies$ $A$ is finite.
        \item $A$ is infinite $\implies$ $B$ is infinite.
        \item $B$ is countable $\implies$ $A$ is countable.
        \item $A$ is uncountable $\implies$ $B$ is uncountable.
    \end{enumerate}
    \textbf{Hint : }Since $f$ is injective, observe the function $h: A \to f(A)$ defined as $h(x) = f(x) \,\,\, \forall x \in A$ is a bijection.
    Hence, $A \sim f(A) \subseteq B$. Now, use the results from Problem 21 on $f(A)$ and $B$.
    \item Suppose $f : A \to B$ is surjective then prove that 
    \begin{enumerate}
        \item $A$ is finite $\implies$ $B$ is finite.
        \item $B$ is infinite $\implies$ $A$ is infinite.
        \item $A$ is countable $\implies$ $B$ is countable.
        \item $B$ is uncountable $\implies$ $A$ is uncountable.
    \end{enumerate}
    \textbf{Hint : }Consider the function $F : B \to A$ defined as,
    $$F(y) = \min \{x : f(x) = y \}$$ 
    $F$ can be shown to be a injection, since $f$ is a function. Then, use the results in Problem 22.
    \item Show that the sets $[0, 1]$ and $(0, 1)$ are equipotent.
    
	    \textbf{Solution : } Let us define a function $f : [0, 1] \to (0, 1)$ as,
	    \begin{equation*}
		    f(x)=
		    \begin{cases}
			    \frac{1}{2} & \text{ if }x = 0 \\
			    \frac{1}{n + 2} & \text{ if } \exists n \in \nn \text{ such that } x = \frac{1}{n} \\
			    x & \text{ otherwise }
		    \end{cases}
	    \end{equation*}
	    $f$ can be shown to a bijection (see the proof in the alternate solution for hints). Hence, 
	    $[0, 1] \sim (0, 1)$

	    \textbf{Alternate Solution : }Let us define $f : (0, 1) \to (0, 1]$, $g : [0, 1) \to [0, 1]$ and $h: (0,1] \to [0, 1)$ as, 
    \begin{equation*}
        f(x) = 
        \begin{cases}
            \frac{1}{n-1} & \text{,if } \exists n \in \nn \text{ such that }x = \frac{1}{n}\\
            x & \text{,if } x \neq \frac{1}{n} \forall n \in \nn \\
        \end{cases}
    \end{equation*}
    \begin{equation*}
        g(x) = 
        \begin{cases}
            \frac{1}{n-1} & \text{,if } \exists n \in \nn \text{ such that }x = \frac{1}{n}\\
            x & \text{,if } x \neq \frac{1}{n} \forall n \in \nn \\
        \end{cases}
    \end{equation*}
    $$h(x) = 1 - x\,\,\, \forall x \in (0,1]$$
    Let $x, y \in (0, 1)$ such that $x \neq y$.
    
    \textbf{Case 1} $x = \frac{1}{m}$ and $y = \frac{1}{n}$ for some $m,n \in \nn$
    \begin{align*}
        x \neq y
        \Rightarrow &\frac{1}{m} \neq \frac{1}{n} \\
        \Rightarrow &m \neq n \\
        \Rightarrow &m - 1 \neq n - 1 \\
        \Rightarrow &\frac{1}{m - 1} \neq \frac{1}{n - 1} \\
        \Rightarrow &f(x) \neq f(y)
    \end{align*}
    \textbf{Case 2} $x \neq \frac{1}{m}, y \neq \frac{1}{n} \,\,\, \forall m,n \in \nn$
        \begin{align*}
            f(x) = x \And f(y) = y\text{ and since, }x \neq y \text{ we have } f(x) \neq f(y)
        \end{align*}
    \textbf{Case 3} $x = \frac{1}{m} \text{ for some } m \in \nn$ and $y \neq \frac{1}{n} \,\,\, \forall n \in \nn$
    \begin{align*}
        f(x) = \frac{1}{n - 1} \And f(y) = y \neq \frac{1}{n - 1} \Rightarrow f(x) \neq f(y)
    \end{align*}
    Hence $f$ is one-one. Now, let $z \in (0, 1]$ be arbitrary
    
    \textbf{Case 1} $z = \frac{1}{n} \text{ for some } n \in \nn$

        $\text{Let us define }x := \frac{1}{n + 1} \text{ then }, f(x) = \frac{1}{n} = z$

    \textbf{Case 2} $z \neq \frac{1}{n} \,\,\,\forall n \in \nn$

        Let us set $x = z$. Then we have $f(x) = x = z$

    Hence $f$ is surjective. Thus, $f$ is a bijection.
    
    Similarly, $g$ can be shown to be a bijection. Also, $h$ is clearly a bijection And thus, the function $F : (0, 1) \to [0, 1]$ defined as, $$F(x) = g\circ h\circ f(x) \, \, \, \forall x \in (0,1)$$ is a bijection. Hence, $(0,1) \sim [0, 1]$ \qed
    
    \textbf{Alternate Solution : }The set $[0, 1]$ can be written as $(0, 1) \cup \{0, 1 \}$. Since $(0, 1)$ is infinite and $\{0, 1\}$ is finite and countable, $(0, 1) \sim (0, 1) \cup \{0, 1\}\Rightarrow (0, 1) \sim [0, 1]$. \qed
    \item Show that the sets $[0, 1]$ and $\p(\nn)$ are equipotent.

	    \textbf{Solution : }Consider the function $f : \p(\nn) \to [0,1]$ deifned as, 
	    $$f(S) = \sum_{n\in S} 10^{-n} \,\,\, \forall S \in \p(\nn)$$ It can be shown that $f$ is injective. 
	    Consider the function $g: [0, 1] \to \p(\nn)$ defined as, 
	    \begin{align*}
	    g(x) = \{\left \lfloor 10^n(x)_2 \right \rfloor : n \in \nn \} 
	    \end{align*}
	    where $(x)_2$ denotes the binary expansion of the number $x$ but with a leading $1$ at the start. 
	    For example, the binary representation of $0.5$ is $0.01111\dots$ 
	    but $(0.5)_2 = 1.011\dots$. This has been done to preserve the information about leading 0's right after the decimal. 
	    For numbers with non-unique decimal expansions (i.e. the numbers of the form $\frac{a}{2^k}$ where $a \in \z$ and $k \in \nn$ 
	    for example, $0.5$ has two binary expansions $0.10000\dots$ and $0.0111\dots$),
	    we choose the representation that ends in infinte 1's. 
	    Since the binary expansion of the numbers is unique, $g$ is also an injection. 

	    Since both $f: \p(\nn) \to [0,1]$ and $g: [0,1] \to \p(\nn)$ are injective, by Schröder–Bernstein theorem, it is possible to find
	    a bijection from $[0, 1] \to \p(\nn)$. Hence, $[0, 1] \sim \p(\nn)$ \qed
	    
	    % \textbf{Alternate Solution : }Consider the function $f:\p(\nn) \to [0,1]$ defined as,
	    % $$f(S) = \sum_{n \in S} 2^{-n} \,\,\, \forall S \in \p(\nn)$$
	    % Since the binary representation of every real number is unique, 
	    % $f$ can be shown to be a bijection. Hence, $[0,1] \sim \p(\nn)$. \qed

    \item Are the sets $\p(\rn)$ and $\p(\nn)$ equipotent? Give justification. 

    \textbf{Solution : }We know from questions 24 and 25, $\p(\nn) \sim [0, 1] \sim (0, 1)$.
    Also consider the function $f : (0, 1) \to \rn$ defined as $f(x) = \tan(\sin^{-1}(2x - 1))$. It can be shown to be a bijection. Hence, $(0, 1) \sim \rn$. Hence, $\p(\nn) \sim \rn$. Now, by Cantor's theorem we know that $P(\rn) \not \sim \rn$. Hence, $\p(\nn) \not\sim \p(\rn)$ \qed
\end{enumerate}
\end{document}
