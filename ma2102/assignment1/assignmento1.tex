\documentclass[10pt]{scrartcl}

\usepackage[T1]{fontenc}
\usepackage{geometry}
\usepackage{graphicx}
\usepackage{amssymb}
\usepackage{amsmath}
\usepackage{enumitem}
\usepackage[english]{babel}
\usepackage{amsthm}
% \usepackage[hidelinks]{hyperref}

\newcommand{\nn}{\mathbb{N}}
\newcommand{\rn}{\mathbb{R}}
\newcommand{\q}{\mathbb{Q}}
\newcommand{\p}{\mathcal{P}}
\newcommand{\z}{\mathbb{Z}}
\title{MA2102 - Linear Algebra I}  
\subtitle{Assignment 1 Solutions}
\author{Debayan Sarkar \\ \texttt{22MS002}}
\date{\today}

\geometry{a4paper, margin=0.8in}
\setlength{\parindent}{0pt}
\begin{document}
\maketitle
\begin{enumerate}
    \item
        \begin{enumerate}[label = {(\roman*)}]
            \item Explain why a system of linear equations with more variables than equations always has a
            solution, whereas a system of such equations with more equations than variables may not have
            any solution at all.
            \item Show that a matrix with more columns than rows (resp. more rows than columns) does not
            have a left (resp. right) inverse.
        \end{enumerate}
    \item Compute the determinant of the following matrix:
        \[
        \begin{pmatrix}
            2  & -2 &          &        &         &   &\\
            -1 &  5 & -2       &        &         &   &\\
               & -2 & 5        & -2     &         &   &  \\
               &    & \ddots   & \ddots & \ddots  &   & \\
               &    &          & -2     & 5       &-2 & \\
               &    &          &        & -2      &5  &-1 \\
               &    &          &        &         &-2 & 2\\
        \end{pmatrix}_{n \times n}
        \]
    \item Let $\lambda$ be an eigenvalue of an $n \times n$ real matrix $A$. Show that there
    exists a positive integer $k \leq n$ such that $$|\lambda - a_{kk}| \leq \sum_{j=1,j \neq k}^n|a_{jk}|$$

    \textbf{Soltion :} First we claim that the eigenvalues of $A$ and $A^T$.
    Hence 
    \item Show that an $n \times n$ real matrix is invertible if and only if its columns
    span $R^n$ .
    \item
    \begin{enumerate}[label={(\roman*)}]
        \item Let $V$ be the set of all real numbers. Define the binary operation
        “addition” on $V$ by $$x \boxplus y = \text{ the maximum of x and y }$$ for all $x, y \in V$ and define an operation of "scalar multiplication" by $$\alpha \boxdot y = \alpha x$$ for all $\alpha \in \rn$ and $x \in V$. Is $V$ a vector space over $\rn$ under the above operations? Justify your answer!
        
        \textbf{Solution : }$V$ is not a vector space over $\rn$ under the defined operations. Let's assume to the contrary, that $V$ is a vector space. Then, $(V, \boxplus)$ must be an abelian group. Consider the element $3 \in V$. According to the definition of $\boxplus$, $3 \boxplus 2 = 3$. Hence, $2 \in V$ is the identity eement in $V$. But, we also have $1 \in V$ satisfying, $3 \boxplus 1 = 3$. Hence, 1 is also an  indentity element in $V$. This is a contradiction, since an abelian group must have a unique identity element. This proves our claim. \qed
        \item Let $V$ be the set of all positive real numbers. Define the binary operation “addition” on $V$ by $$x \boxplus y = xy$$ for all $x, y \in V$. Define an operation of "scalar multiplication" by $$\alpha \boxdot x = x^\alpha$$ for all $\alpha \in \rn$ and $x \in V$. Show that $V$ is a vector space over $\rn$. Provide a basis for V.

        \textbf{Solution : } $V$ is a vector space over $\rn$ under the defined binary operations. 

        We first show that $(V, \boxplus)$ is an abelian group. Let $x, y, z \in V$ be arbitrary. 

        Then, if $z := x \boxplus y = xy > 0$. Hence, $z \in R^+ = V$ Hence, $V$ is closed under $\boxplus$.

        Also, $x \boxplus (y \boxplus z) = x \boxplus (yz) = x(yz) = (xy)z = (xy) \boxplus z = (x \boxplus y) \boxplus z$
        Since multiplication is associative in $\rn$. Hence, $\boxplus$ is associative in $V$. 
        
        Consider the element $1 \in V$. Then, we have $1 \boxplus x = 1x = x = x\cdot1 = x \boxplus 1$. Hence $1$ is the identity element in V.
        Now we show tha uniqueness of the identity element. Let's assume there's another identiy element $\bar{1} \in V$
        Then, we have $1 = 1 \boxdot \bar{1} = \bar{1}$. Hence, $V$ has a unique identity element under $\boxplus$.

        We know that $\exists \text{ a unique } x^{-1} \in \rn^+ = V$ such that, $x \boxplus x^{-1} = x \cdot x^{-1} = 1$ which is the identity element.
        Hence, each element in $V$ has a unique inverse under $\boxplus$.
 
        Now, we also have $x \boxplus y = xy = yx = y \boxplus x$ since multiplication is commutative
        in $\rn$. Hence, $\boxplus$ is commutative ion $V$.
        
        This proves that $(V, \boxplus)$ is an abelian group.

        Now let $\alpha, \beta \in \rn$ and $u, v \in V$ be arbitrary.
        
        Note that,
        \begin{enumerate}[label = {(\alph*)}]
            \item $w := \alpha \boxdot u = u^\alpha > 0 \Rightarrow w \in \rn^+ = V$
            \item $1 \boxdot v = v^1 = v$
            \item $\alpha \boxdot (\beta \boxdot v) = \alpha \boxdot v^\beta = (v^\beta)^\alpha =v^{\beta \alpha} = v^{\alpha \beta} =  (v^\alpha)^\beta = \beta \boxdot (v^\alpha) = \beta \boxdot (\alpha \boxdot v)$
            \item $(\alpha + \beta)v = v^{\alpha + \beta} = v^\alpha \cdot v^\beta = v^\alpha \boxplus v^\beta = \alpha \boxdot v \boxplus \beta \boxdot v$
            \item $\alpha \boxdot (u \boxplus v) = \alpha \boxdot uv = (uv)^\alpha = u^\alpha \cdot v^\alpha = \alpha \boxdot u \boxplus \alpha \boxdot v$
        \end{enumerate}

        Hence, $V$ is a vector space over $\rn$.  \qed

        We can take $2 \in V$ as a basis for $V$ since $2^\alpha$ is a injective continous function from $\rn \to \rn^+$ Thus, every $y \in V$ will have a unique $\alpha$ such that, $\alpha \boxdot x = x^\alpha = y$.
    \end{enumerate}
\end{enumerate}
\end{document}

