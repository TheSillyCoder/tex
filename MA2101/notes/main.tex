\documentclass{scrartcl}
\usepackage{amsmath,amsfonts, amssymb}
\usepackage{amsthm}
\usepackage[english]{babel}
\usepackage{enumitem}
\begin{document}

\newcommand{\rn}{\mathbb{R}}
\newtheorem{remark}{Remark}

\title{MA2101: Analysis I Lecture Notes}
\subtitle{Instructor: Rajib Dutta}
\author{Sabarno Saha \\ \texttt{22MS037}\and Debayan Sarkar \\ \texttt{22MS002}}
\date{August 13, 2023}
    \maketitle
    % \tableofcontents
    \section{Algebra of the Real Number System}
    \subsection{Properties of Addition}
    The properties of addition(+) in the real number system are:
    \begin{enumerate}[label={(A\arabic*})]
        \item $x+y=y+x\; \forall \; x,y \in \mathbb{R}$
        \item $(x+y)+z = x+(y+z) \; \forall \; x,y,z \in \mathbb{R}$
        \item $ \exists !0 \in \mathbb{R} \; s.t.\; x+0 =0+x =x \; \forall \; x\in \mathbb{R}$
        \item $ \forall x \in \rn \exists! y \in \rn \text{ s.t. } x + y = y + x = 0$
    \end{enumerate}
    \subsection{Properties of Multiplication}
    The properties of multiplication($\cdot$) in the real number system are:
    \begin{enumerate}[label={(M\arabic*)}]
        \item $x\cdot y = y\cdot x \forall x, y \in \rn$ 
        \item $(x \cdot y)\cdot z = x\cdot (y \cdot z) \forall x, y, z \in \rn$
        \item $\exists ! 1 \in \rn \text{ s.t. } x \cdot 1 = x \forall x \in \rn$
        \item $\forall x\in \rn \setminus \{0\} \exists ! y \in \rn \text{ s.t. } x\cdot y = y \cdot x = 0$
    \end{enumerate}
    \subsection{Distributive Property}
    The multiplication operator distributes over addition inn real numbers.
    $$x \cdot (y + z) = x \cdot y + x \cdot z$$
    Since addition and multiplication have these properties in real numbers, $(\rn, +, \cdot)$ is a Field.
    \subsection{Order in Reals}
    \subsubsection{Law of Trichotomy}
    Given two $x, y \in \rn$, exaclt one of the following statements is true :
    \begin{enumerate}[label={(\roman*)}]
        \item $x = y$
        \item $x > y$
        \item $x < y$
    \end{enumerate}
    \subsubsection{Properties of "$<$"}
    \begin{enumerate}[label={(\roman*)}]
        \item If $x < y$ and $y < z$ then $x < z$
        \item If $x > 0$, $y>0$ then, $xy >0$
        \item If $x < y$ then, $x + z < y +z$ $\forall z \in \rn$
        \item $x < y \Rightarrow -x > -y$
        \item If $x < y \text{ and } z > 0 \text{ then } xz < yz$
        \item If $0<x<y$, then $0<\frac{1}{y}<\frac{1}{x}$
        \item $x^2 \geq 0 \forall x \in \rn$
    \end{enumerate}
    \begin{remark}
        Let $x, y \in \rn$ such that, $x \leq y$ and $y \leq x$. Then, $x=y$. 
        \begin{proof}
            Let's assume to teh contrary that $x \neq y$. Then, by the law of trichotomy, either $x < y$ or $y < x$.
            Let $y < x$. From $x \leq y$ we have either $x < y$ or $x = y$. By the law of trichotomy, neither of them can be true. Hence, $y \nless x$
            Now, let $x < y$. From $y \leq x$ we have either $y < x$ or $y = x$. Again, by the law of trichotomy, neither of them can be true. Hence, $x \nless y$
            This is a contradiction. Hence, $x = y$
        \end{proof}
    \end{remark}
\end{document}
