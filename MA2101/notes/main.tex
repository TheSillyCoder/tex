\documentclass{scrartcl}
\usepackage{handout}
\usepackage{amsmath,amsfonts, amssymb}
\usepackage{amsthm}
\usepackage[english]{babel}
\usepackage{enumitem}

\newcommand{\rn}{\mathbb{R}}
\newcommand{\nn}{\mathbb{N}}
\newcommand{\q}{\mathbb{Q}}
\newcommand{\p}{\mathcal{P}}
\newcommand{\z}{\mathbb{Z}}
\newcommand{\lub}{\text{lub}}

\begin{document}
    \title{MA2101: Analysis I Lecture Notes}
    \subtitle{Instructor: Rajib Dutta}
    \author{
        Debayan Sarkar
            \thanks{{\href{https://thesillycoder.github.io}{TheSillyCoder.github.io}}}
            \\\href{mailto:ds22ms002@iiserkol.ac.in}{22MS002} \and
        Sabarno Saha
        % \thanks{{}}
        \\\href{mailto:ss22ms037@iiserkol.ac.in}{22MS037} \and
        Piyush Kumar Singh
            \thanks{{\href{https://iampiyushkrsingh.github.io}{iamPiyushKrSingh.github.io}}}
            \\\href{mailto:pks22ms027@iiserkol.ac.in}{22MS027}
    }
    \date{August 13, 2023}
    \maketitle
    \tableofcontents
    \section{Algebra of the Real Number System}
    \subsection{Properties of Addition}
    The properties of addition(+) in the real number system are:
    \begin{enumerate}[label={(A\arabic*})]
        \item $x+y=y+x\; \forall \; x,y \in \mathbb{R}$
        \item $(x+y)+z = x+(y+z) \; \forall \; x,y,z \in \mathbb{R}$
        \item $ \exists !0 \in \mathbb{R} \; s.t.\; x+0 =0+x =x \; \forall \; x\in \mathbb{R}$
        \item $ \forall x \in \rn \exists! y \in \rn \text{ s.t. } x + y = y + x = 0$
    \end{enumerate}
    \subsection{Properties of Multiplication}
    The properties of multiplication($\cdot$) in the real number system are:
    \begin{enumerate}[label={(M\arabic*)}]
        \item $x\cdot y = y\cdot x \forall x, y \in \rn$ 
        \item $(x \cdot y)\cdot z = x\cdot (y \cdot z) \forall x, y, z \in \rn$
        \item $\exists ! 1 \in \rn \text{ s.t. } x \cdot 1 = x \forall x \in \rn$
        \item $\forall x\in \rn \setminus \{0\} \exists ! y \in \rn \text{ s.t. } x\cdot y = y \cdot x = 0$
    \end{enumerate}
    \subsection{Distributive Property}
    The multiplication operator distributes over addition inn real numbers.
    $$x \cdot (y + z) = x \cdot y + x \cdot z$$
    Since addition and multiplication have these properties in real numbers, $(\rn, +, \cdot)$ is a Field.
    \subsection{Order in Reals}
    \subsubsection{Law of Trichotomy}
    Given two $x, y \in \rn$, exactly one of the following statements is true :
    \begin{enumerate}[label={(\roman*)}]
        \item $x = y$
        \item $x > y$
        \item $x < y$
    \end{enumerate}
    \subsubsection{Properties of \texorpdfstring{$<$}{lt}}
    \begin{enumerate}[label={(\roman*)}]
        \item If $x < y$ and $y < z$ then $x < z$
        \item If $x > 0$, $y>0$ then, $xy >0$
        \item If $x < y$ then, $x + z < y +z$ $\forall z \in \rn$
        \item $x < y \Rightarrow -x > -y$
        \item If $x < y \text{ and } z > 0 \text{ then } xz < yz$
        \item If $0<x<y$, then $0<\frac{1}{y}<\frac{1}{x}$
        \item $x^2 \geq 0 ~ \forall x \in \rn$
    \end{enumerate}
    \begin{remark}
        Let $x, y \in \rn$ such that, $x \leq y$ and $y \leq x$. Then, $x=y$. 
        \begin{proof*}
            Let's assume to teh contrary that $x \neq y$. Then, by the law of trichotomy, either $x < y$ or $y < x$.
            Let $y < x$. From $x \leq y$ we have either $x < y$ or $x = y$. By the law of trichotomy, neither of them can be true. Hence, $y \nless x$
            Now, let $x < y$. From $y \leq x$ we have either $y < x$ or $y = x$. Again, by the law of trichotomy, neither of them can be true. Hence, $x \nless y$
            This is a contradiction. Hence, $x = y$
        \end{proof*}
    \end{remark}
	\begin{example}
        For $x<y$, we have, $x<\frac{x+y}{2}<y$. The point $\frac{x + y}{2}$ is called the midpoint between $x$ and $y$.
        \begin{proof*}
            Since $x < y$, we have $\frac{x}{2} < \frac{y}{2}$. Then, we have $\frac{x}{2} + \frac{x}{2} < \frac{y}{2} + \frac{x}{2} \Rightarrow x < \frac{x + y}{2}$
		    Similarly, we have $\frac{x}{2} + \frac{y}{2} < \frac{y}{2} + \frac{y}{2}\Rightarrow \frac{x + y}{2} < y$
            Hence, $x < \frac{x + y}{2} < y$
        \end{proof*}
	\end{example}
    \begin{example}
        If $x \leq y + z$ for all $z > 0$, then $x \leq y$.
        \begin{proof*}
            Let $x,y \in \rn$ such that $x \leq y + z$ for all $z > 0$. We claim that, $x \leq y$. Let us assume to the contrary that, $x > y$. Then, we have $x - y > 0$. Let $\epsilon := x - y$. Also observe that, $x-y \leq z$ for all $z > 0$. Let us set $z = \frac{\epsilon}{2}$. Then, $x - y \leq z \Rightarrow \epsilon \leq \frac{\epsilon}{2} \Rightarrow 1 \leq \frac{1}{2}$. This is a contradiction. Hence, $x \leq y$. This proves our claim.
        \end{proof*}
    \end{example}
    \begin{example}
        For $0 < x < y$, we have $0<x^2<y^2$ and $0<\sqrt{2}<\sqrt{y}$, assuming te existence of $\sqrt{x}$ and $\sqrt{y}$. More generally, if $x$ and $y$ are positive, then $x < y \text{ iff } x^n < y^n$ for all $n \in \nn$.
		\begin{proof*}
            will type up later
		\end{proof*}
    \end{example}
    \begin{example}
        For $0 < x < y$, we have $\sqrt{xy} < \frac{x + y}{2}$.
        \begin{proof*}
            We claim that the statment is true. Let us assume to the contrary that, $\frac{x + y}{2} < \sqrt{xy}$. Then, we have, 
            \begin{align*}
			    &\frac{x + y}{2} < \sqrt{xy} \\
			    \Rightarrow &\left(\frac{x + y}{2}\right)^2 < xy \tag{Example 1}\\
			    \Rightarrow &\left(\frac{x + y}{2}\right)^2 - xy < 0 \\
			    \Rightarrow &\left(\frac{x - y}{2}\right)^2 < 0 \\
		    \end{align*}
		    This is a contradiction since we know that $\alpha^2 \geq 0 \,\,\, \forall \alpha \in \rn$. This proves our claim.
        \end{proof*}
    \end{example}

    \section{Upper and Lower Bounds}
    \begin{definition}[Upper Bound]
        Let $A \subset \rn$ be nonempty. A number $\alpha \in \rn$ is said to be the upper bound of $A$ if $\forall x \in A$, we have $x \leq \alpha$
    \end{definition}
    Geometrically, this means that on the real number line, all the elements of $A$ are to the left of $\alpha$. If $\alpha \in \rn$ is not an upper bound of $A$, then $\exists x \in A \text{ s.t. } x > \alpha$
    \begin{definition}[Lower Bound]
        Let $A \subset \rn$ be nonempty. A number $\alpha \in \rn$ is said to be the lower bound of $A$ if $\forall x \in A$, we have $x \geq \alpha$
    \end{definition}
    Geometrically, this means that on the real number line, all the elements of $A$ are to the right of $\alpha$. If $\alpha \in \rn$ is not an lower bound of $A$, then $\exists x \in A \text{ s.t. } x < \alpha$
    \begin{example}
        Consider the set $A := \{1, 2, 3, 4, 5\}$ Then the upper bounds of this $A$ are $5, 6, 1729 \dots$ etc.
        And the lower bound of $A$ are $1, 0, -1 \dots$ etc. Hence, lower and upper bounds of a set are not unique.
    \end{example}
    \begin{definition}[Bounded above set]
        Let $\phi \neq A \subset \rn$. Then, $A$ is said to be bounded above, if $\exists \alpha \in \rn$ such that $\alpha$ is an upper bound of A.
    \end{definition}
    \begin{definition}[Bounded below set]
        Let $\phi \neq A \subset \rn$. Then, $A$ is said to be bounded below, if $\exists \alpha \in \rn$ such that $\alpha$ is an lower bound of A.
    \end{definition}
    \begin{theorem}
        Let $A$ be a bounded above set. Let $\alpha \in \rn$ be an upper bound. Let $\beta \in \rn$ such that 
        $\beta \geq \alpha$. Then $\beta$ is an upperbound of $A$.
        \begin{proof*}
            Since $\alpha$ is an upper bound, we have $x \leq \alpha \,\, \forall x \in A$
            But, $\alpha \leq \beta$. Then we have, $x \leq \beta \,\, \forall x \in A$.
            Hence $\beta$ is an upper bound of A.
        \end{proof*}
    A similar theorem can be stated and proved analagously for bounded below sets and lower bounds.
    \end{theorem}
    \begin{definition}[Maximum]
        Let $\phi \neq A \subset \rn$. $\alpha \in \rn$ is said to be the maximum of $A$ if 
        \begin{enumerate}[label={(\roman*)}]
            \item $\alpha \in \rn$
            \item $\alpha$ is an upper bound of $A$ 
        \end{enumerate}
        The maximum of a set $A$ is denoted by $\max\{A\}$
    \end{definition}
    \begin{theorem}
        The maximum of a set is unique.
        \begin{proof*}
            Let $\alpha$ and $\beta$ be two maxima of $A$ s.t. $\alpha \neq \beta$. Then, by the law of trichotomy,
            either $\alpha > \beta$ or $\beta > \alpha$. 
            If $\alpha > \beta$, then $\alpha$ cannot be an upper bound since, $\beta \in A$. And,
            if $\beta > \alpha$, then $\beta$ cannot be an upper bound since, $\alpha \in A$. This is a contradiction. 
            Hence, $\alpha = \beta$.
        \end{proof*}
    \end{theorem}
    \begin{remark}
        A bounded above set need not have a maximum. Consider set $S : = (0,1)$. $S$ clearly has upper bounds
        for instance $1$, $2$, etc. However it does not have a maximum, since none of the upper bounds are in the set $S$ itself. Some more of such remarks could be
        \begin{enumerate}
            \item $\{ 1 - \frac{1}{n} : n \in \nn\}$
            \item $\{1 - \frac{1}{2^n} : n \in \nn \}$
        \end{enumerate}
    \end{remark}
    \begin{definition}[Least Upper Bound]
        Let $A$ be a non-empty subset of $\rn$. An upper bound $\alpha$ is said to be the least upper bound of $A$
        if $\beta < \alpha \Rightarrow \beta \text{ is not an upper bound of A.}$

        We denote the least upper bound of $A$ as $\text{lub}A$
    \end{definition}
    \begin{definition}[Greatest Lower Bound]
        Given a non-empty bounded below set A, a real number $\beta$ is said to be the greatest lower bound of A if 
        \begin{enumerate}[label={(\roman*)}]
            \item $\beta$ is a lower bound of $A$
            \item if $\beta' > \beta$ then, $\exists x \in A$ such that $\beta \leq x \leq \beta'$
        \end{enumerate}

        We denote the Greatest lower bound of $A$ as $\text{glb}A$
    \end{definition}
    \begin{theorem}
        The least upper bound(or greatest lower bound) of a bounded above(or below) set is unique.
    \end{theorem}
    \begin{definition}[Completeness Axiom]
        Every bounded above(or below) subset of $\rn$ has a least upper bound(or greatest lower bound).
    \end{definition}
    \begin{remark}    
        This is an important property in $\rn$. $\q$ does not have this property. For instance, consider 
        the set $A = \{ x\in \q : 0<x^2<2\}$. Clearly, $\text{lub}A = \sqrt{2} \notin \q$
    \end{remark}
    \begin{theorem}[Archimedean Property]
        (AP1) $\nn$ is not bounded above in $\rn$ \\
        (AP2) Let $x>0$, $y \in \rn$, then $\exists n \in \nn$ such that $nx > y$
        \begin{proof*}
            First, we prove AP1 by contradiction. Let us assume to the contrary that $\nn$ is bounded above
            in $\rn$. Then, by the Completeness Axiom, $\exists \alpha \in \rn$ such that $\alpha = \lub \nn$. Then, by definition of lub, 
            $\alpha -1$ is not an upper bound of $\nn$. Hence, $\exists n \in \nn \text{ such that } \alpha -1<n<\alpha$. Then 
            we have $n +1 > \alpha$. This is a contradiction since $n + 1 \in \nn$ and $\alpha$ is an upper bound of $\nn$. Hence, $\nn$ is not bounded above
            in $\rn$.

            Now, we show that AP1 $\Rightarrow$ AP2. However, that trivially follows, since if we have $x, y \in \rn$ with 
            $x > 0$, then from $AP1$ we have that $\frac{y}{x}$ cannot be an upper bound of $\rn$. Hence, 
            $\exists n \in \nn \text{ s.t. } n > \frac{y}{x}$, i.e. $nx > y$.

            The proof for AP2 $\Rightarrow$ AP1 is left as an exercise.
        \end{proof*}
    \end{theorem}
    \begin{definition}[Bounded Sets]
        Let $A$ be a non-empty subset of $\rn$. Then, $A$ is said to be bounded if $A$ is bounded above and bounded below.
    \end{definition}
    \begin{theorem}[Greatest Integer Function]
        Let $x \in \rn$. Then, $\exists$ a unique $m \in \z$ s.t. $m \leq x < m+1$
        This $m$ is denoted as $\floor{x}$. We define the \textbf{greatest integer function} $f : \rn \to \z$ as, $f(x) = \floor{x}$
    \end{theorem}
\end{document}
