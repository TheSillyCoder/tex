\documentclass[10pt]{scrartcl}

\usepackage[T1]{fontenc}
\usepackage{amssymb, amsmath, amsthm}
\usepackage{geometry, graphicx, enumitem, wrapfig, fancyhdr,cancel, physics}
\usepackage[english]{babel}

% \usepackage{listings, xcolor}
% \definecolor{codegreen}{rgb}{0,0.6,0}
% \definecolor{codegray}{rgb}{0.5,0.5,0.5}
% \definecolor{codepurple}{rgb}{0.58,0,0.82}
% \definecolor{backcolour}{rgb}{0.95,0.95,0.92}

% \lstdefinestyle{mystyle}{
%     backgroundcolor=\color{backcolour},   
%     commentstyle=\color{codegreen},
%     keywordstyle=\color{magenta},
%     numberstyle=\tiny\color{codegray},
%     stringstyle=\color{codepurple},
%     basicstyle=\ttfamily\footnotesize,
%     breakatwhitespace=false,         
%     breaklines=true,                 
%     captionpos=b,                    
%     keepspaces=true,                 
%     numbers=left,                    
%     numbersep=5pt,                  
%     showspaces=false,                
%     showstringspaces=false,
%     showtabs=false,                  
%     tabsize=4
% }
% \lstset{style=mystyle}
\geometry{a4paper, margin=0.8in}
\pagestyle{fancy}
\lhead{PH3102 - Assignment 2 Solutions}
\rhead{Debayan Sarkar \texttt{22MS002}}
\everymath{\displaystyle}
\theoremstyle{definition}
\newtheorem{exercise}{Question}
\newenvironment{solution} {\begin{proof}[\normalfont \textbf{Solution}]} {\end{proof}}

\renewcommand{\qedsymbol}{}
\newcommand{\nn}{\mathbb{N}}
\newcommand{\npixL}{\frac{n\pi x}{L}}
\newcommand{\rn}{\mathbb{R}}
\newcommand{\q}{\mathbb{Q}}
\newcommand{\p}{\mathcal{P}}
\newcommand{\z}{\mathbb{Z}}
\newcommand{\dx}{\mathrm{d}x}
\newcommand*{\OO}{\hat{O}}
\newcommand*{\Op}{\hat{p}}
\newcommand*{\Ox}{\hat{x}}
\newcommand*{\OH}{\hat{H}}
\newcommand*{\Oa}{\hat{a}}
\title{PH3102 - Quantum Mechanics}  
\subtitle{Assignment 2 Solutions}
\author{Debayan Sarkar \\ \texttt{22MS002}}
\date{\today}

\geometry{a4paper, margin=1in}
\setlength{\parindent}{0pt}
\begin{document}
\maketitle
\begin{exercise}\textbf{Concept of an operator. [5 Marks]}

    Consider $\hat{O}$ to be an operator defined by $$\hat{O} = \ket{\phi}\bra{\psi},$$ where $\ket{\phi}$ and $\ket{\psi}$ are 
two vectors of the state space.
    \begin{enumerate}[label={(\alph*)}]
        \item Give the condition for $\hat{O}$ to be Hermitian.
        \item Calculate $\hat{O}^2$. State the condition for $\hat{O}$ to be a projection operator.
        \item Show that $\hat{O}$ can always be written in the form of $\hat{O} = \lambda P_1 P_2 $, where $\lambda$
            is a constant and $P_1$ and $P_2$ are projection operators corresponding to the vectors $\ket{\phi}$ and $\ket{\psi}$
            respectively.
    \end{enumerate}
\end{exercise}
 
\begin{solution}
    $ $
    \begin{enumerate}[label={(\alph*)}]
        \item For $\hat{O}$ to be Hermitian, we must have 
            \begin{align*}
                &\hat{O} = \hat{O}^\dagger \\ 
                \Rightarrow &\ket{\phi} \bra{\psi} = (\ket{\phi} \bra{\psi})^\dagger \\
                \Rightarrow &\ket{\phi} \bra{\psi} = \ket{\psi} \bra{\phi} \\
                \Rightarrow &\ket{\phi} \bra{\psi} \ket{\psi} = \ket{\psi} \bra{\phi} \ket{\psi} \tag{Acting on $\ket{\psi}$}\\
                \Rightarrow &\ket{\phi} \bra{\psi} \ket{\psi} = \cfrac{\ip{\phi}{\psi}}{\ip{\psi}{\psi}} \ket{\psi}\\
                \Rightarrow &\ket{\phi} = c\ket{\psi} \\
            \end{align*}
            Where $c = \cfrac{\ip{\phi}{\psi}}{\ip{\psi}{\psi}}$. Now we have, 
            \begin{align*}
                &\hat{O} = \hat{O}^\dagger \\ 
                \Rightarrow &\ket{\phi} \bra{\psi} = (\ket{\phi} \bra{\psi})^\dagger \\
                \Rightarrow &\ket{\phi} \bra{\psi} = \ket{\psi} \bra{\phi} \\
                \Rightarrow &c^*\ket{\psi} \bra{\psi} = c\ket{\psi} \bra{\psi} \tag{$\ket{\phi} = c\ket{\psi}$}\\
                \Rightarrow &c^* = c \\ 
                \Rightarrow &c \in \rn
            \end{align*}
            Hence, for $\OO$ to be Hermitian we must meet the following conditions $$\boxed{\ket{\phi} = c\ket{\psi}, ~ c \in \rn}$$
        \item We first calculate $\OO^2$. 
            \begin{align*}
                &\OO^2 = \op{\phi}{\psi} \cdot \op{\phi}{\psi} = \ket{\phi} \ip{\psi}{\phi} \bra{\psi} = \ip{\psi}{\phi} \OO \\ 
                \Rightarrow ~ &\boxed{\OO^2 = \ip{\psi}{\phi} \OO}
            \end{align*}
            Hence, for $\OO$ to be a projection operator, we must have $\OO^2 = \OO$. Thus we must have 
            $$\boxed{\ip{\phi}{\psi} = 1}$$
        \item We are given that $P_1 = \op{\phi}$ and $P_2 = \op{\psi}{\psi}$ Then, we have
            \begin{align*}
                &P_1P_2 = \op{\phi}{\phi} \cdot \op{\psi}{\psi} \\
                \Rightarrow &P_1P_2 = \ket{\phi}\ip{\phi}{\psi}\bra{\psi} \\
                \Rightarrow &P_1P_2 = \ip{\phi}{\psi} \ket{\phi}\bra{\psi} \\
                \Rightarrow &\cfrac{P_1P_2}{\ip{\phi}{\psi}} = \OO \tag{Assuming that $\ip{\phi}{\psi} \neq 0$}\\
                \Rightarrow &\OO = \lambda P_1 P_2
            \end{align*}
            Where $\lambda = \cfrac{1}{\ip{\phi}{\psi}}$. Observe that, if $\ip{\phi}{\psi} = 0$, $P_1P_2 = 0$. Hence, we will not
            be able to find a lambda such that $\OO = \lambda P_1P_2$. 
    \end{enumerate}
\end{solution}
\begin{exercise}\textbf{Characteristics of a real wavefunction. [5 Marks]}

    Consider a real-valued wavefunction $\psi(x)$.
    \begin{enumerate}[label=(\alph*)]
        \item For this $\psi(x)$, show that the expectation value of momentum given by $\expval{\Op}$ is zero.

        \item Now show that if $\psi(x)$ has a mean momentum given by $\expval{\Op}$, $e^{\flatfrac{i p_0 x}{\hbar}} \psi(x)$ has mean momentum $\expval{\Op} + p_0$.
    \end{enumerate}
    Use the Dirac “bra-ket” notation to carry out the computations.
\end{exercise}
\begin{solution}
    $ $
    \begin{enumerate}[label=(\alph*)]
        \item We first calculate $\expval{\Op}$ as
            \begin{align*}
                \expval{\Op} &= \ev{\Op}{\psi} \\ 
                             &= \int_{-\infty}^{\infty} \int_{-\infty}^{\infty} \ip{\psi}{x'} \mel{x'}{\Op}{x} \ip{x}{\psi} dx'dx \\ 
                             &= \int_{-\infty}^{\infty} \int_{-\infty}^{\infty} \psi(x') \qty(-i\hbar \delta(x' - x)\dv{x}) \psi(x) dx'dx \tag{$\psi^*(x') = \psi(x')$}\\ 
                             &= \int_{-\infty}^{\infty} \psi(x) \qty(-i\hbar \dv{x}) \psi(x) dx \\ 
                             &= -i\hbar \int_{-\infty}^{\infty} \psi(x) \dv{\psi(x)}{x} dx \tag{i}\\ 
                             &= -i\hbar \qty[ \eval{\psi^2(x)}_{-\infty}^{\infty} - \int_{-\infty}^{\infty} \psi(x) \dv{\psi(x)}{x} dx] \\ 
                             &= i\hbar \int_{-\infty}^{\infty} \psi(x) \dv{\psi(x)}{x} dx  \tag{Since $\psi(x)$ must vanish at $\pm \infty$}\\
                             &= -\expval{\Op} \tag{using (i)} \\
            \end{align*}
            Hence, we have $$\expval{\Op} = - \expval{\Op} \Rightarrow \boxed{\expval{\Op} = 0}$$
        \item Let $\phi(x) = e^{\flatfrac{ip_0x}{\hbar}}\psi(x)$. Then we have,
            \begin{align*}
                \ev{\Op}{\phi}  &= \int_{-\infty}^{\infty} \int_{-\infty}^{\infty} \ip{\phi}{x'} \mel{x'}{\Op}{x} \ip{x}{\phi} dx'dx \\ 
                                &= \int_{-\infty}^{\infty} \int_{-\infty}^{\infty} \phi^*(x') \qty(-i\hbar \delta(x - x')\dv{x}) \phi(x) dx'dx \\ 
                                &= \int_{-\infty}^{\infty} \int_{-\infty}^{\infty} e^{-\flatfrac{ip_0x'}{\hbar}}\psi(x') \qty(-i\hbar \delta(x - x')\dv{x}) e^{\flatfrac{ip_0x}{\hbar}}\psi(x) dx'dx \\ 
                                &= \int_{-\infty}^{\infty}  e^{-\flatfrac{ip_0x}{\hbar}}\psi(x) \qty(-i\hbar \dv{x}) e^{\flatfrac{ip_0x}{\hbar}}\psi(x) dx \\ 
                                &= -i\hbar\int_{-\infty}^{\infty}  e^{-\flatfrac{ip_0x}{\hbar}}\psi(x) \dv{x}e^{\flatfrac{ip_0x}{\hbar}}\psi(x) dx \\ 
                                &= -i\hbar\qty[\int_{-\infty}^{\infty}  e^{-\flatfrac{ip_0x}{\hbar}}\psi(x) e^{\flatfrac{ip_0x}{\hbar}}\dv{\psi(x)}{x} dx + \int_{-\infty}^{\infty}  e^{-\flatfrac{ip_0x}{\hbar}}\psi(x) \cfrac{ip_0}{\hbar}e^{\flatfrac{ip_0x}{\hbar}}\psi(x) dx] \\ 
                                &= -i\hbar\qty[\int_{-\infty}^{\infty}  \psi(x) \dv{\psi(x)}{x} dx + \cfrac{ip_0}{\hbar}\int_{-\infty}^{\infty} \psi^2(x) dx] \\ 
                                &= -i\hbar\int_{-\infty}^{\infty}  \psi(x) \dv{\psi(x)}{x} dx + p_0 \tag{Assuming $\psi(x)$ is normalized}\\ 
                                &= \expval{\Op} + p_0 \tag{using (i)}
            \end{align*}
            Hence we have, $$\boxed{\ev{\Op}{\phi} = \expval{\Op} + p_0}$$.

    \end{enumerate}
\end{solution}
\begin{exercise}\textbf{Coherent States. [5 Marks]}
    For the simple harmonic oscillator with the time-independent wavefunctions $\psi_n(x)$ satisfying
    \begin{equation*}
        \OH \psi_n(x) = \hbar \omega \qty(n + \cfrac{1}{2}) \psi_n(x),
    \end{equation*}
    consider the superposition at time $t = 0$
    \begin{equation*}
        \psi(x, t = 0) = \sum_{n=0}^\infty c_n \psi_n(x).
    \end{equation*}
    \begin{enumerate}[label=(\alph*)]
        \item How should the coefficients be chosen so that $\psi(x, 0)$ is an eigenstate of lowering operator $\hat{a}$ with eigenvalue $\alpha$ (a given complex number), i.e.,
              \begin{equation*}
                  \hat{a} \psi(x, 0) = \alpha \psi(x, 0).
              \end{equation*}

        \item Using the expression for $\hat{a}$, find the explicit form of the wavefunction at $\psi(x, 0)$. Ensure that $\psi(x, 0)$ is correctly normalized.
    \end{enumerate}
    Note that eigenstates of $\hat{a}$ are referred to as "coherent states".
\end{exercise}
\begin{solution}
    Let us define the kets $\ket{n}$ such that, $$\psi_n(x) = \ip{\Ox}{n}$$ We know that the raising
    and lowering operators are given by,
    \begin{align*}
        \Oa &= \sqrt{\cfrac{m\omega}{2\hbar}}\qty(\Ox + \cfrac{i}{m\omega}\Op) \\
        \Oa^\dagger &= \sqrt{\cfrac{m\omega}{2\hbar}}\qty(\Ox - \cfrac{i}{m\omega}\Op)
    \end{align*}
    And, we know that 
    \begin{align*}
        \Oa \ket{n} &= \sqrt{n}\ket{n - 1} \\
        \Oa^\dagger \ket{n} &= \sqrt{n + 1}\ket{n + 1}
    \end{align*}
    \begin{enumerate}[label=(\alph*)]
        \item We are given that $\Oa \ket{\psi} = \alpha \ket{\psi}$. Where $\ip{\Ox}{\psi} = \psi(x, t = 0)$ For that to hold, we must have, 
            \begin{align*}
                &\Oa\ket{\psi} = \sum_{n = 0}^{\infty} c_n \Oa \ket{n} \\
                \Rightarrow &\alpha \ket{\psi} = \sum_{n = 1}^{\infty} c_n \sqrt{n} \ket{n - 1} \\
                \Rightarrow &\sum_{n = 0}^{\infty} \alpha c_n \ket{n} = \sum_{n = 0}^{\infty} c_{n + 1} \sqrt{n + 1} \ket{n} \\ 
                \Rightarrow &\alpha c_n = \sqrt{n + 1} c_{n + 1} \\
                \Rightarrow &c_{n + 1} = \cfrac{\alpha}{\sqrt{n + 1}} c_{n}\\
                \Rightarrow &c_{n} = \cfrac{\alpha}{\sqrt{n}}c_{n - 1} \\
                \Rightarrow &c_{n} = \cfrac{\alpha^2}{\sqrt{n(n-1)}} c_{n - 2}\\
                            &\quad \quad \quad \quad ~~ \vdots \\
                \Rightarrow &\boxed{c_n = \cfrac{\alpha^n}{\sqrt{n!}}c_0}
            \end{align*}
            Then our state $\ket{\psi}$ becomes, $$\boxed{\ket{\psi} = c_0 \sum_{n = 0}^{\infty} \cfrac{\alpha^n}{\sqrt{n!}} \ket{n}}$$
            For $\ket\psi$ to be normalized, we must have,
            \begin{align*}
                &\ip{\psi}{\psi} = 1 \\ 
                \Rightarrow &|c_0|^2 \sum_{n = 0}^{\infty} \sum_{m = 0}^{\infty} \cfrac{(\alpha^*)^n \cdot \alpha^m}{\sqrt{n! \cdot m!}} \ip{n}{m}  = 1\\
                \Rightarrow &|c_0|^2 \sum_{n = 0}^{\infty} \sum_{m = 0}^{\infty} \cfrac{(\alpha^*)^n \cdot \alpha^m}{\sqrt{n! \cdot m!}} \delta{nm}  = 1\\
                \Rightarrow &|c_0|^2 \sum_{n = 0}^{\infty} \cfrac{|\alpha|^{2n}}{n!} = 1\\
                \Rightarrow &|c_0|^2 e^{|\alpha|^2} = 1 \\ 
                \Rightarrow &|c_0|^2 = e^{-|\alpha|^2} \\
                \Rightarrow &\boxed{c_0 = \exp\qty(i\phi - \cfrac{|\alpha|^2}{2})}
            \end{align*}
            Where $\phi \in \rn$ is a constant. Hence finally our state $\psi$ turns out to be, 
            $$\boxed{\ket\psi = \exp\qty(i\phi - \cfrac{|\alpha|^2}{2}) \sum_{n = 0}^{\infty} \cfrac{\alpha^n}{\sqrt{n!}}\ket n}$$
            And in position representation we have,
            $$\boxed{\psi(x, t = 0) = \exp\qty(i\phi - \cfrac{|\alpha|^2}{2}) \sum_{n = 0}^{\infty} \cfrac{\alpha^n}{\sqrt{n!}} ~ \psi_n(x)}$$
        \item We are given that, $\Oa \psi = \alpha \psi$. Then in the position representation we have,
            \begin{align*}
                \alpha \psi(x) &= \mel{x}{\Oa}{\psi}\\ 
                               &= \int_0^\infty \int_0^\infty \ip{x}{x'}\mel{x'}{\Oa}{x''}\ip{x''}{\psi} dx'' dx' \\ 
                               &= \sqrt{\cfrac{m\omega}{2\hbar}}\int_0^\infty \int_0^\infty \ip{x}{x'}\mel{x'}{\Ox + \cfrac{i}{m\omega}\Op}{x''}\ip{x''}{\psi} dx'' dx' \\ 
                               &= \sqrt{\cfrac{m\omega}{2\hbar}}\int_0^\infty \int_0^\infty \delta(x - x')\qty(\delta(x'-x'')x'' + \cfrac{i}{m\omega}\cdot \qty(-i\hbar) \delta(x'-x'')\dv{x''})\psi(x'') dx'' dx' \\ 
                               &= \sqrt{\cfrac{m\omega}{2\hbar}}\int_0^\infty \delta(x -x')\qty(x' + \cfrac{\hbar}{m\omega}\dv{x'})\psi(x') dx' \\ 
                               &= \sqrt{\cfrac{m\omega}{2\hbar}}\qty(x + \cfrac{\hbar}{m\omega}\dv{x})\psi(x) \\ 
            \end{align*}
            Hence, we arrive at a first order differential equation in $\psi(x)$.
            \begin{align*}
                &\sqrt{\cfrac{m\omega}{2\hbar}}\qty(x + \cfrac{\hbar}{m\omega} \dv{x})\psi = \alpha \psi(x) \\
                \Rightarrow &\dv{\psi}{x} = \cfrac{m\omega}{\hbar}\qty(\sqrt{\cfrac{2\hbar}{m\omega}}\alpha - x)\psi(x) \\
                \Rightarrow &\boxed{\dv{\psi}{x} = \qty(\sqrt{\cfrac{2m\omega}{\hbar}}\alpha - \cfrac{m\omega x}{\hbar})\psi(x)} \\
            \end{align*}
            Observe that, the following guess solution for $\psi(x)$ works
            $$\boxed{\psi(x) = C\exp\qty(\sqrt{\cfrac{2m\omega}{\hbar}}~\alpha x - \cfrac{m\omega x^2}{2\hbar})}$$
            \begin{align*}
                \dv{\psi}{x} &= C\qty(\sqrt{\cfrac{2m\omega}{\hbar}}~\alpha - \cfrac{m\omega x}{\hbar}) \exp\qty(\sqrt{\cfrac{2m\omega}{\hbar}}~\alpha x - \cfrac{m\omega x^2}{2\hbar}) \\ 
                             &= \qty(\sqrt{\cfrac{2m\omega}{\hbar}}~\alpha - \cfrac{m\omega x}{\hbar})\psi(x)
            \end{align*}
            Now we must ensure that $\psi(x)$ is normalized. Then, we must have, 
            \begin{align*}
                &\int_{-\infty}^\infty \psi^*(x)\psi(x)dx = 1 \\ 
                \Rightarrow &|C|^2 \int_{-\infty}^\infty \exp\qty(\sqrt{\cfrac{2m\omega}{\hbar}}2\Re(\alpha)x - \cfrac{m\omega x^2}{\hbar} )dx  = 1\\ 
                \Rightarrow &|C|^2 \int_{-\infty}^\infty \exp\qty(-\qty(\sqrt{\cfrac{m\omega}{\hbar}}x - \sqrt 2\Re(\alpha))^2 + 2\Re(\alpha)^2)dx  = 1\\ 
                \Rightarrow &|C|^2 \exp\qty(2\Re(\alpha)^2) \qty(\cfrac{\hbar}{m\omega})^{\flatfrac{1}{2}} \sqrt\pi = 1 \\
                \Rightarrow &|C|^2 = \qty(\cfrac{m\omega}{\pi\hbar})^{\flatfrac{1}{2}} \exp\qty(-2\Re(\alpha)^2) \\
                \Rightarrow &\boxed{C = \qty(\cfrac{m\omega}{\pi\hbar})^{\flatfrac{1}{4}} \exp\qty(i\theta -\Re(\alpha)^2)}
            \end{align*}
            Where $\theta \in \rn$ is a constant. Hence, our normaized state is given by, 
            $$\boxed{\psi(x, t=0) = \qty(\cfrac{m\omega}{\pi\hbar})^{\flatfrac{1}{4}}\exp\qty(i\theta -\Re(\alpha)^2 + \sqrt{\cfrac{2m\omega}{\hbar}}~\alpha x - \cfrac{m\omega x^2}{2\hbar})}$$
    \end{enumerate}
\end{solution}
\end{document}
